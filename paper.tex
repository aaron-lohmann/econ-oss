% Options for packages loaded elsewhere
\PassOptionsToPackage{unicode}{hyperref}
\PassOptionsToPackage{hyphens}{url}
\PassOptionsToPackage{dvipsnames,svgnames,x11names}{xcolor}
%
\documentclass[
  12pt,
]{article}

\usepackage{amsmath,amssymb}
\usepackage{iftex}
\ifPDFTeX
  \usepackage[T1]{fontenc}
  \usepackage[utf8]{inputenc}
  \usepackage{textcomp} % provide euro and other symbols
\else % if luatex or xetex
  \usepackage{unicode-math}
  \defaultfontfeatures{Scale=MatchLowercase}
  \defaultfontfeatures[\rmfamily]{Ligatures=TeX,Scale=1}
\fi
\usepackage{lmodern}
\ifPDFTeX\else  
    % xetex/luatex font selection
\fi
% Use upquote if available, for straight quotes in verbatim environments
\IfFileExists{upquote.sty}{\usepackage{upquote}}{}
\IfFileExists{microtype.sty}{% use microtype if available
  \usepackage[]{microtype}
  \UseMicrotypeSet[protrusion]{basicmath} % disable protrusion for tt fonts
}{}
\makeatletter
\@ifundefined{KOMAClassName}{% if non-KOMA class
  \IfFileExists{parskip.sty}{%
    \usepackage{parskip}
  }{% else
    \setlength{\parindent}{0pt}
    \setlength{\parskip}{6pt plus 2pt minus 1pt}}
}{% if KOMA class
  \KOMAoptions{parskip=half}}
\makeatother
\usepackage{xcolor}
\setlength{\emergencystretch}{3em} % prevent overfull lines
\setcounter{secnumdepth}{1}
% Make \paragraph and \subparagraph free-standing
\makeatletter
\ifx\paragraph\undefined\else
  \let\oldparagraph\paragraph
  \renewcommand{\paragraph}{
    \@ifstar
      \xxxParagraphStar
      \xxxParagraphNoStar
  }
  \newcommand{\xxxParagraphStar}[1]{\oldparagraph*{#1}\mbox{}}
  \newcommand{\xxxParagraphNoStar}[1]{\oldparagraph{#1}\mbox{}}
\fi
\ifx\subparagraph\undefined\else
  \let\oldsubparagraph\subparagraph
  \renewcommand{\subparagraph}{
    \@ifstar
      \xxxSubParagraphStar
      \xxxSubParagraphNoStar
  }
  \newcommand{\xxxSubParagraphStar}[1]{\oldsubparagraph*{#1}\mbox{}}
  \newcommand{\xxxSubParagraphNoStar}[1]{\oldsubparagraph{#1}\mbox{}}
\fi
\makeatother


\providecommand{\tightlist}{%
  \setlength{\itemsep}{0pt}\setlength{\parskip}{0pt}}\usepackage{longtable,booktabs,array}
\usepackage{calc} % for calculating minipage widths
% Correct order of tables after \paragraph or \subparagraph
\usepackage{etoolbox}
\makeatletter
\patchcmd\longtable{\par}{\if@noskipsec\mbox{}\fi\par}{}{}
\makeatother
% Allow footnotes in longtable head/foot
\IfFileExists{footnotehyper.sty}{\usepackage{footnotehyper}}{\usepackage{footnote}}
\makesavenoteenv{longtable}
\usepackage{graphicx}
\makeatletter
\def\maxwidth{\ifdim\Gin@nat@width>\linewidth\linewidth\else\Gin@nat@width\fi}
\def\maxheight{\ifdim\Gin@nat@height>\textheight\textheight\else\Gin@nat@height\fi}
\makeatother
% Scale images if necessary, so that they will not overflow the page
% margins by default, and it is still possible to overwrite the defaults
% using explicit options in \includegraphics[width, height, ...]{}
\setkeys{Gin}{width=\maxwidth,height=\maxheight,keepaspectratio}
% Set default figure placement to htbp
\makeatletter
\def\fps@figure{htbp}
\makeatother
% definitions for citeproc citations
\NewDocumentCommand\citeproctext{}{}
\NewDocumentCommand\citeproc{mm}{%
  \begingroup\def\citeproctext{#2}\cite{#1}\endgroup}
\makeatletter
 % allow citations to break across lines
 \let\@cite@ofmt\@firstofone
 % avoid brackets around text for \cite:
 \def\@biblabel#1{}
 \def\@cite#1#2{{#1\if@tempswa , #2\fi}}
\makeatother
\newlength{\cslhangindent}
\setlength{\cslhangindent}{1.5em}
\newlength{\csllabelwidth}
\setlength{\csllabelwidth}{3em}
\newenvironment{CSLReferences}[2] % #1 hanging-indent, #2 entry-spacing
 {\begin{list}{}{%
  \setlength{\itemindent}{0pt}
  \setlength{\leftmargin}{0pt}
  \setlength{\parsep}{0pt}
  % turn on hanging indent if param 1 is 1
  \ifodd #1
   \setlength{\leftmargin}{\cslhangindent}
   \setlength{\itemindent}{-1\cslhangindent}
  \fi
  % set entry spacing
  \setlength{\itemsep}{#2\baselineskip}}}
 {\end{list}}
\usepackage{calc}
\newcommand{\CSLBlock}[1]{\hfill\break\parbox[t]{\linewidth}{\strut\ignorespaces#1\strut}}
\newcommand{\CSLLeftMargin}[1]{\parbox[t]{\csllabelwidth}{\strut#1\strut}}
\newcommand{\CSLRightInline}[1]{\parbox[t]{\linewidth - \csllabelwidth}{\strut#1\strut}}
\newcommand{\CSLIndent}[1]{\hspace{\cslhangindent}#1}

\usepackage[noblocks]{authblk}
\renewcommand*{\Authsep}{, }
\renewcommand*{\Authand}{, }
\renewcommand*{\Authands}{, }
\renewcommand\Affilfont{\small}
        \usepackage[margin=1in]{geometry}  % Sets 1-inch margins
        \usepackage[dvipsnames]{xcolor}
% \usepackage[top=1in, bottom=1in, left=0.8in, right=0.8in]{geometry}
\makeatletter
\@ifpackageloaded{caption}{}{\usepackage{caption}}
\AtBeginDocument{%
\ifdefined\contentsname
  \renewcommand*\contentsname{Table of contents}
\else
  \newcommand\contentsname{Table of contents}
\fi
\ifdefined\listfigurename
  \renewcommand*\listfigurename{List of Figures}
\else
  \newcommand\listfigurename{List of Figures}
\fi
\ifdefined\listtablename
  \renewcommand*\listtablename{List of Tables}
\else
  \newcommand\listtablename{List of Tables}
\fi
\ifdefined\figurename
  \renewcommand*\figurename{Figure}
\else
  \newcommand\figurename{Figure}
\fi
\ifdefined\tablename
  \renewcommand*\tablename{Table}
\else
  \newcommand\tablename{Table}
\fi
}
\@ifpackageloaded{float}{}{\usepackage{float}}
\floatstyle{ruled}
\@ifundefined{c@chapter}{\newfloat{codelisting}{h}{lop}}{\newfloat{codelisting}{h}{lop}[chapter]}
\floatname{codelisting}{Listing}
\newcommand*\listoflistings{\listof{codelisting}{List of Listings}}
\makeatother
\makeatletter
\makeatother
\makeatletter
\@ifpackageloaded{caption}{}{\usepackage{caption}}
\@ifpackageloaded{subcaption}{}{\usepackage{subcaption}}
\makeatother
\ifLuaTeX
  \usepackage{selnolig}  % disable illegal ligatures
\fi
\usepackage{bookmark}

\IfFileExists{xurl.sty}{\usepackage{xurl}}{} % add URL line breaks if available
\urlstyle{same} % disable monospaced font for URLs
\hypersetup{
  pdftitle={Open sourcing the economics of open source: A comprehensive literature review},
  pdfauthor={aaron-lohmann},
  colorlinks=true,
  linkcolor={NavyBlue},
  filecolor={Maroon},
  citecolor={NavyBlue},
  urlcolor={Blue},
  pdfcreator={LaTeX via pandoc}}

\title{Open sourcing the economics of open source: A comprehensive
literature review\thanks{Comments, questions, and feedback are welcome
and can be submitting as an issue to the repository
\href{https://github.com/aaron-lohmann/econ-oss/issues}{https://github.com/aaron-lohmann/econ-oss}.
Alternatively, to:
\href{aaron.lohmann@uni-bielefeld.de}{aaron.lohmann@uni-bielefeld.de}.}}


  \author{aaron-lohmann}
  
\date{2025-04-02}
\begin{document}
\maketitle
\begin{abstract}
Open source software (OSS) is powering much of our digital
infrastructure. From hosting websites, to working with databases and
advancement of Machine Learning models. Recently, OSS has received
increased attention from the economic discipline. Some work is motivated
by the curious incentive structures, the econmic value created through
OSS or the available and detailed micro-data. The goal for this project
is to openly and publically craft a literature review which summarises
our findings about OSS relating to economic ideas. Inspired by OSS
itself, this project is open to contribution from anyone. Reward in the
form of co-authorship is given according to contribution (see
contribution guidelines).
\end{abstract}

\begin{center}
\bf{Contributions welcome!}
\end{center}

Contribution guidelines:

\begin{enumerate}
\def\labelenumi{\arabic{enumi}.}
\tightlist
\item
  Contributions to this project must be made through a pull request to
  the repo \url{https://github.com/aaron-lohmann/econ-oss}.
\item
  Position in author listing will be determind by the amount of approved
  pull requests. Ties between the same amount of pull requests will be
  broken by the earliest contribution of an author.
\item
  Co-authorship will be granted upon two approved pull requets. One of
  these contributions must introduce a new citation.
\end{enumerate}

\newpage

\section{Introduction}\label{sec-introduction}

Open Source Software (OSS) is key to the functioning of the modern
digital infrastructure. For economists some parts of OSS propose curious
questions. Why do developers contribute? Can open development outpeform
closed development? On the other hand, the detailed micro data on
individuals allows to study questions that are relevant to a number of
knowledge intensive sectors like patenting, R \& D and the development
of Work from Home. OSS provides the rare oppurunity to track for a large
group of workers who works when for how long at what time. Moreover,
there are questions related to global integration. How do teams form
across space, how much do teams across collaborate with each other? Yet,
other another aspect for OSS is the network structure. One example is to
study the adoption of technology. This review tries to be comprehensive.
From descriptive papers to published datasets and novel and innovati ve
insights. In the end, this paper addresses two groups of readers. Those
that read for knowledge and look for answers and those that read for
understanding, potentially seeking to enter the field of economics of
OSS.

\section{Theory}\label{sec-theory}

Theoretical contributions which directly answer questions about OSS.

\section{Datasets}\label{sec-datasets}

What datasets are out there to study OSS? What are benefits of one
versus the other?

Schueller et al. (\citeproc{ref-schueller2022}{2022}) provide detailed
data on one particular ecosystem: Rust and its associated package
manager crates.io. The data covers a time period from early \(2014\) all
the way till September \(2022\). Given the relative youth of Rust, this
can be seen as representative of the full history of the language. The
data covers commits, issues, pull requests, importantly dependency
relationships betwenn packages. Unique in this list of datasets, the
authors also provide data from other code hosting platforms outside
GitHub: Gitlab, Bitbucket and the likes. This dataset is well suited for
more concentrated deep micro data work given the careful construction.

\section{The value of OSS}\label{sec-value}

Hoffmann, Nagle, and Zhou (\citeproc{ref-hoffmann2024value}{2024}) is an
effort in measuring the value of OSS. The authors identify \(2,000\)
(\(1,000\) from NPM and \(1,000\) from non-NPM) key packages in the OSS
infrastructure. For those they map supply side costs of creation and a
demand side value based on firms and websites using these packages.
Their preferred estimate for the supply side value is \(4.15\) billion
USD and demand side value greater than \(8.8\) trillion USD.

\section{Geographical composition and team formation}\label{sec-teams}

Goldbeck (\citeproc{ref-goldbeck2025bit}{2025}) concentrats on software
developers in the U.S. Using the self-reported locations, develoepers
are matched to 179 economic areas. Running gravity-style regressions,
the author finds sizable colocation effects. However, this effect is
much smaller than that of traditional inventor networks as for example
in patenting. For developers with joint organizational membership or
those collaborting on high quality projects, measured by stars, the
colocation effect is found to be weaker.

\section{Team outcomes}\label{sec-outcomes}

Betti et al. (\citeproc{ref-betti2024dynamics}{2024}) study team
dynamics with a special emphasis on leaders. Builiding on the data by
Schueller et al. (\citeproc{ref-schueller2022}{2022}) they identify
leaders of projects and reveal that often the workload is highly
heterogeneous. In teams of \(5\) to \(7\), the leader roughly does
\(80\) per cent of the commits. More hetereogenous effort distribution
is correlated with higher success. Finally, they study changes in
leaderships and reveal that subsequent to a change in leadership the
success of the project increases.

\section{References}\label{sec-references}

\phantomsection\label{refs}
\begin{CSLReferences}{1}{0}
\bibitem[\citeproctext]{ref-betti2024dynamics}
Betti, Lorenzo, Luca Gallo, Johannes Wachs, and Federico Battiston.
2024. {``The Dynamics of Leadership and Success in Software Development
Teams.''} \emph{arXiv Preprint arXiv:2404.18833}.

\bibitem[\citeproctext]{ref-goldbeck2025bit}
Goldbeck, Moritz. 2025. {``Bit by Bit: Colocation and the Death of
Distance in Software Developer Networks.''} \emph{Journal of Economic
Geography}, lbaf002.

\bibitem[\citeproctext]{ref-hoffmann2024value}
Hoffmann, Manuel, Frank Nagle, and Yanuo Zhou. 2024. {``The Value of
Open Source Software.''} \emph{Harvard Business School Strategy Unit
Working Paper}, no. 24-038.

\bibitem[\citeproctext]{ref-schueller2022}
Schueller, William, Johannes Wachs, Vito D. P. Servedio, Stefan Thurner,
and Vittorio Loreto. 2022. {``Evolving Collaboration, Dependencies, and
Use in the Rust Open Source Software Ecosystem.''} \emph{Scientific
Data} 9 (1): 703. \url{https://doi.org/10.1038/s41597-022-01819-z}.

\end{CSLReferences}

\setcounter{section}{0}
\renewcommand{\thesection}{\Alph{section}}

\setcounter{table}{0}
\renewcommand{\thetable}{A\arabic{table}}

\setcounter{figure}{0}
\renewcommand{\thefigure}{A\arabic{figure}}



\end{document}
